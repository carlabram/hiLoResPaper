% mnras_template.tex
%
% LaTeX template for creating an MNRAS paper
%
% v3.0 released 14 May 2015
% (version numbers match those of mnras.cls)
%
% Copyright (C) Royal Astronomical Society 2015
% Authors:
% Keith T. Smith (Royal Astronomical Society)

% Change log
%
% v3.0 May 2015
%    Renamed to match the new package name
%    Version number matches mnras.cls
%    A few minor tweaks to wording
% v1.0 September 2013
%    Beta testing only - never publicly released
%    First version: a simple (ish) template for creating an MNRAS paper

%%%%%%%%%%%%%%%%%%%%%%%%%%%%%%%%%%%%%%%%%%%%%%%%%%
% Basic setup. Most papers should leave these options alone.
\documentclass[a4paper,fleqn,usenatbib]{mnras}

% MNRAS is set in Times font. If you don't have this installed (most LaTeX
% installations will be fine) or prefer the old Computer Modern fonts, comment
% out the following line
\usepackage{newtxtext,newtxmath}
% Depending on your LaTeX fonts installation, you might get better results with one of these:
%\usepackage{mathptmx}
%\usepackage{txfonts}

% Use vector fonts, so it zooms properly in on-screen viewing software
% Don't change these lines unless you know what you are doing
\usepackage[T1]{fontenc}
%\usepackage{ae,aecompl}


%%%%% AUTHORS - PLACE YOUR OWN PACKAGES HERE %%%%%

% Only include extra packages if you really need them. Common packages are:
\usepackage{graphicx}	% Including figure files
\usepackage{amsmath}	% Advanced maths commands
\usepackage{amssymb}	% Extra maths symbols
\usepackage{color}

%%%%%%%%%%%%%%%%%%%%%%%%%%%%%%%%%%%%%%%%%%%%%%%%%%

%%%%% AUTHORS - PLACE YOUR OWN COMMANDS HERE %%%%%

\newcommand{\Mpc}{{\rm Mpc}}
\newcommand{\km}{{\rm km}}
\newcommand{\kpc}{{\rm kpc}}
\newcommand{\pc}{\ {\rm pc}}
\newcommand{\kms}{{\rm km}\,{\rm s}^{-1}}
\newcommand{\yr}{{\rm yr}}
\newcommand{\Msun}{{\rm M}_\odot}
\newcommand{\Mstel}{M_\ast}
\newcommand{\logM}{\log\Mstel/\Msun}
\newcommand{\LCDM}{$\Lambda$CDM}
\newcommand{\resp}{respectively}
\newcommand{\bfr}{\bf\color{red}}
\newcommand{\bfb}{\color{myblue}}
\newcommand{\bfnull}{\color{black}}
\newcommand{\bfc}{\sf\color{myblue}}
\newcommand{\bfp}{\bf\color{magenta}}
\newcommand{\ssfr}{{\rm sSFR}}
\newcommand{\sfr}{{\rm SFR}}
\newcommand{\zphot}{z_{\rm phot}}
\newcommand{\zspec}{z_{\rm spec}}

\newcommand{\beq}{\begin{equation}}
\newcommand{\eeq}{\end{equation}}
\newcommand{\bitem}{\begin{itemize}}
\newcommand{\eitem}{\end{itemize}}
\newcommand{\benum}{\begin{enumerate}}
\newcommand{\eenum}{\end{enumerate}}

\mathchardef\mhyphen="2D

\newcommand{\ntot}{{\bfr XXX}} % total objects in high-res sample
\newcommand{\midz}{{\bfr ZZZ}} % their median redshift

\newcommand{\facilities}{{\it Facilities:}}
\newcommand{\software}{{\it Software:}}

% Please keep new commands to a minimum, and use \newcommand not \def to avoid
% overwriting existing commands. Example:
%\newcommand{\pcm}{\,cm$^{-2}$}	% per cm-squared

%%%%%%%%%%%%%%%%%%%%%%%%%%%%%%%%%%%%%%%%%%%%%%%%%%

%%%%%%%%%%%%%%%%%%% TITLE PAGE %%%%%%%%%%%%%%%%%%%

% Title of the paper, and the short title which is used in the headers.
% Keep the title short and informative.
\title[Who needs spectra?]{Spectral resolution is not important for modeling galaxy growth}

% The list of authors, and the short list which is used in the headers.
% If you need two or more lines of authors, add an extra line using \newauthor
\author[L.E.~Abramson et al.]{L.E.~Abramson$^{1}$\thanks{E-mail: \href{mailto:labramson@carnegiescience.edu}{labramson@carnegiescience.edu}}, 
D.D.~Kelson$^{1}$
\\
\\
% List of institutions
$^1$	Carnegie Observatories, 813 Santa Barbara Street, Pasadena, CA 91101, USA\\
}

% These dates will be filled out by the publisher
\date{Accepted XXX. Received YYY; in original form ZZZ}
%\date{Submitted to {\it MNRAS} 31 May 2019}

% Enter the current year, for the copyright statements etc.
\pubyear{2020}

% Don't change these lines
\begin{document}
\label{firstpage}
\pagerange{\pageref{firstpage}--\pageref{lastpage}}
\maketitle

% Abstract of the paper
\begin{abstract}


\end{abstract}
%{\it Hubble Frontier Fields} and {\it Cluster Lensing and Supernova with Hubble}

% Select between one and six entries from the list of approved keywords.
% Don't make up new ones.
\begin{keywords}
	galaxies: surveys --- galaxies: spectroscopy --- spectroscopy: techniques
\end{keywords}

%%%%%%%%%%%%%%%%%%%%%%%%%%%%%%%%%%%%%%%%%%%%%%%%%%

%%%%%%%%%%%%%%%%% BODY OF PAPER %%%%%%%%%%%%%%%%%%

\section{Introduction}
\label{sec:intro}

One of the main ambitions of the study of galaxy evolution is to understand how galaxies grow in 
stellar mass over time. This is done by decomposing galaxies' spectral energy distributions (SEDs)
into combinations of stellar populations of different ages. The coefficients that weight each stellar
population---which have distinct colors---then represent the amount of stellar mass a galaxy is
inferred to have formed at the lookback time that corresponds to that age.

The SEDs of stellar populations are distinct but not orthogonal. As such, there are degeneracies.
Obviously, those are compounded by formally age-independent effects from metallicity and dust
reddening. To alleviate those degeneracies---especially with respect to metallicity---and increase
the contrast of the extant stellar populations, high resolution spectra ($R\sim1000$s) can be used in 
addition to sets of broadband colors. The hope in this approach is that the details of absorption 
lines that are differentially sensitive to age and metallicity might enable more accurate coefficients
to be inferred for the amount of mass formed at any given lookback time.

The utility of this approach is testable. For example, spectral decompositions of low spectral resolution
SEDs---comprised of photometry or a mix of photometry and $R\lesssim100s$ spectroscopy---can
be modeled to infer a star formation history (SFH), which can in turn be used to produce a prediction of
features at quasi-arbitrarily higher spectral resolution. These predictions can be compared to 
actual high resolution observations to see whether said features add any useful age-sensitive
information that might enhance the accuracy of the inferred SFHs.

We carry that experiment out here. Using a set of \ntot



%------------------------------------------------------------------------------------------------------------------------------------------
%------------------------------------------------------------------------------------------------------------------------------------------

\section{Data}
\label{sec:data}


%------------------------------------------------------------------------------------------------------------------------------------------
%------------------------------------------------------------------------------------------------------------------------------------------

\section{Sample Characteristics}
\label{sec:sampChar}


%------------------------------------------------------------------------------------------------------------------------------------------
%------------------------------------------------------------------------------------------------------------------------------------------

\section{Discussion}
\label{sec:discussion}


%------------------------------------------------------------------------------------------------------------------------------------------
%------------------------------------------------------------------------------------------------------------------------------------------

\section{Summary}
\label{sec:summary}

Foo.\\

\noindent\facilities\ Magellan/IMACS

\noindent\software\ IDL (Coyote libraries; \url{http://www.idlcoyote.com/}), python ({\texttt CarPy}).

%------------------------------------------------------------------------------------------------------------------------------------------
%------------------------------------------------------------------------------------------------------------------------------------------

\section*{Acknowledgements}


%------------------------------------------------------------------------------------------------------------------------------------------
%------------------------------------------------------------------------------------------------------------------------------------------

%%%%%%%%%%%%%%%%%%%% REFERENCES %%%%%%%%%%%%%%%%%%

% The best way to enter references is to use BibTeX:

\bibliographystyle{mnras}
\bibliography{/Users/labramson/lit} % if your bibtex file is called example.bib


% Alternatively you could enter them by hand, like this:
% This method is tedious and prone to error if you have lots of references
%\begin{thebibliography}{99}
%\bibitem[\protect\citeauthoryear{Author}{2012}]{Author2012}
%Author A.~N., 2013, Journal of Improbable Astronomy, 1, 1
%\bibitem[\protect\citeauthoryear{Others}{2013}]{Others2013}
%Others S., 2012, Journal of Interesting Stuff, 17, 198
%\end{thebibliography}

%%%%%%%%%%%%%%%%%%%%%%%%%%%%%%%%%%%%%%%%%%%%%%%%%%

%%%%%%%%%%%%%%%%% APPENDICES %%%%%%%%%%%%%%%%%%%%%

%\appendix
%\label{sec:appendix}

%%%%%%%%%%%%%%%%%%%%%%%%%%%%%%%%%%%%%%%%%%%%%%%%%%


% Don't change these lines
\bsp	% typesetting comment
\label{lastpage}
\end{document}

% End of mnras_template.tex