% mnras_template.tex
%
% LaTeX template for creating an MNRAS paper
%
% v3.0 released 14 May 2015
% (version numbers match those of mnras.cls)
%
% Copyright (C) Royal Astronomical Society 2015
% Authors:
% Keith T. Smith (Royal Astronomical Society)

% Change log
%
% v3.0 May 2015
%    Renamed to match the new package name
%    Version number matches mnras.cls
%    A few minor tweaks to wording
% v1.0 September 2013
%    Beta testing only - never publicly released
%    First version: a simple (ish) template for creating an MNRAS paper

%%%%%%%%%%%%%%%%%%%%%%%%%%%%%%%%%%%%%%%%%%%%%%%%%%
% Basic setup. Most papers should leave these options alone.
\documentclass[a4paper,fleqn,usenatbib]{mnras}

% MNRAS is set in Times font. If you don't have this installed (most LaTeX
% installations will be fine) or prefer the old Computer Modern fonts, comment
% out the following line
\usepackage{newtxtext,newtxmath}
% Depending on your LaTeX fonts installation, you might get better results with one of these:
%\usepackage{mathptmx}
%\usepackage{txfonts}

% Use vector fonts, so it zooms properly in on-screen viewing software
% Don't change these lines unless you know what you are doing
\usepackage[T1]{fontenc}
%\usepackage{ae,aecompl}


%%%%% AUTHORS - PLACE YOUR OWN PACKAGES HERE %%%%%

% Only include extra packages if you really need them. Common packages are:
\usepackage{graphicx}	% Including figure files
\usepackage{amsmath}	% Advanced maths commands
\usepackage{amssymb}	% Extra maths symbols
\usepackage{color}

%%%%%%%%%%%%%%%%%%%%%%%%%%%%%%%%%%%%%%%%%%%%%%%%%%

%%%%% AUTHORS - PLACE YOUR OWN COMMANDS HERE %%%%%

\newcommand{\Mpc}{{\rm Mpc}}
\newcommand{\km}{{\rm km}}
\newcommand{\kpc}{{\rm kpc}}
\newcommand{\pc}{\ {\rm pc}}
\newcommand{\kms}{{\rm km}\,{\rm s}^{-1}}
\newcommand{\yr}{{\rm yr}}
\newcommand{\Msun}{{\rm M}_\odot}
\newcommand{\Mstel}{M_\ast}
\newcommand{\logM}{\log\Mstel/\Msun}
\newcommand{\LCDM}{$\Lambda$CDM}
\newcommand{\resp}{respectively}
\newcommand{\bfr}{\bf\color{red}}
\newcommand{\bfb}{\color{myblue}}
\newcommand{\bfnull}{\color{black}}
\newcommand{\bfc}{\sf\color{myblue}}
\newcommand{\bfp}{\bf\color{magenta}}
\newcommand{\ssfr}{{\rm sSFR}}
\newcommand{\sfr}{{\rm SFR}}
\newcommand{\tobs}{t_{\rm obs}}
\newcommand{\zphot}{z_{\rm phot}}
\newcommand{\zspec}{z_{\rm spec}}

\newcommand{\beq}{\begin{equation}}
\newcommand{\eeq}{\end{equation}}
\newcommand{\bitem}{\begin{itemize}}
\newcommand{\eitem}{\end{itemize}}
\newcommand{\benum}{\begin{enumerate}}
\newcommand{\eenum}{\end{enumerate}}

\mathchardef\mhyphen="2D

\newcommand{\ntot}{{\bfr XXX}} % total objects in high-res sample
\newcommand{\midz}{{\bfr ZZZ}} % their median redshift

\newcommand{\CITE}{{\bfr CITE}}
\newcommand{\facilities}{{\it Facilities:}}
\newcommand{\software}{{\it Software:}}

% Please keep new commands to a minimum, and use \newcommand not \def to avoid
% overwriting existing commands. Example:
%\newcommand{\pcm}{\,cm$^{-2}$}	% per cm-squared

%%%%%%%%%%%%%%%%%%%%%%%%%%%%%%%%%%%%%%%%%%%%%%%%%%

%%%%%%%%%%%%%%%%%%% TITLE PAGE %%%%%%%%%%%%%%%%%%%

% Title of the paper, and the short title which is used in the headers.
% Keep the title short and informative.
\title[Who needs spectra?]{Spectral resolution is not important for modeling galaxy growth}

% The list of authors, and the short list which is used in the headers.
% If you need two or more lines of authors, add an extra line using \newauthor
\author[Abramson, Kelson, \& Dressler]{Louis E.~Abramson$^{1}$\thanks{E-mail: \href{mailto:labramson@carnegiescience.edu}{labramson@carnegiescience.edu}},
Daniel D.~Kelson$^{1}$,
and Alan Dressler$^{1}$
\\
\\
% List of institutions
$^1$	Carnegie Observatories, 813 Santa Barbara Street, Pasadena, CA 91101, USA\\
}

% These dates will be filled out by the publisher
\date{Accepted XXX. Received YYY; in original form ZZZ}
%\date{Submitted to {\it MNRAS} 31 May 2019}

% Enter the current year, for the copyright statements etc.
\pubyear{2020}

% Don't change these lines
\begin{document}
\label{firstpage}
\pagerange{\pageref{firstpage}--\pageref{lastpage}}
\maketitle

% Abstract of the paper
\begin{abstract}

	We compare actual $R\sim{\bfr 800}$ spectroscopy to model predictions based on galaxy 
	star formation histories (SFHs) inferred from much lower resolution data: $ugrizJK_{s}$ 
	photometry and $R\sim25$ rest-optical prism spectra. From \ntot\ systems, we find a median 
	difference of $\leq$1\% between all predicted and measured absorption features in the Lick index 
	bandpasses except the Blamer lines---explainable by unmodeled emission---and Ca4227 and 
	Fe5270 in {\it UVJ}-classified passive galaxies, which are 1.7\%--2.5\% weaker than expected. 
	{\bfr $\chi^{2}$ stuff.} As such, absent a Ca-- or Fe--age prior accurate to the 2\% level---whose 
	empirical apprehension is itself a motivation for SED fitting---we find no utility in adding 
	high resolution spectroscopy as an SFH modeling constraint, at least when using models that capture 
	the intrinsic diversity of real growth trajectories. Our results cast doubt on the extent to which 
	spectra from the {\it James Webb Space Telescope} will enhance our understanding of galaxy 
	growth and suggest that progress requires new tactics as much as new data.
%	or a $\sigma_{v}-\Mstel$ prior with less than $\sim$0.2	dex of scatter
%	By increasing the number of data points without adding meaningful physical details, such data 
%	may in fact harm our understanding by unrealistically shrinking SFH uncertainties. 
	%, if not the fundamental
%	utility of the project of inferring SFHs from galaxy-level data.

\end{abstract}

% Select between one and six entries from the list of approved keywords.
% Don't make up new ones.
\begin{keywords}
	galaxies: spectroscopy
\end{keywords}

%%%%%%%%%%%%%%%%%%%%%%%%%%%%%%%%%%%%%%%%%%%%%%%%%%

%%%%%%%%%%%%%%%%% BODY OF PAPER %%%%%%%%%%%%%%%%%%

\section{Introduction}
\label{sec:intro}

A central ambition of the study of galaxy evolution is to understand stellar mass growth; i.e., galaxy
star formation histories (SFHs). Spectral energy distributions (SEDs) are the key data in this work because 
they can be decomposed into combinations of distinct stellar subpopulations of known ages. The resulting 
coefficients yield the amount of stellar mass a galaxy is inferred to have formed at the lookback time 
corresponding to each subpopulation's age.
	
Different stellar subpopulations have different but not orthogonal SEDs. As such, galaxy decompositions 
are degenerate. Of course, those degeneracies are compounded by age-independent effects like metallicity 
and dust. 

High resolution spectra ($R\sim500$--5000) are often used to alleviate those degeneracies in SFH model 
fitting. The hope is that the absorption features that emerge in those data will increase the contrast 
between constituent stellar subpopulations, constrain metallicities, and yield more accurate 
age/mass coefficients. The utility of these data is usually taken as axiomatic, but it is also testable. 

Here we present experiment that shows there is in fact little information in high resolution spectra
that enhance constraints on galaxy SFHs compared to inferences based on a combination of broadband 
photometry and low resolution ($R\sim25$) prism spectra.\footnote{A future paper will extend this statement 
to pure photometry-based inferences.} We use precomputed SFH inferences based on such low resolution 
SEDs for a set of \ntot\ systems at $\langle z\rangle=\midz$ to produce predictions 
of each galaxy's high-resolution spectrum. We then compare those predictions to actual high resolution 
($R\sim800$) observations taken post-facto. With the exception of the Balmer lines---whose divergence 
from predictions is readily ascribable to emission line infilling---we find differences to be of order 
{\bfr whatever they are}, suggesting {\bfr whatever we say they do}.

Section \ref{sec:data} describes the data on which our experiment is based, Section \ref{sec:results} 
shows the comparisons between our spectral predictions and the high resolution data, and Section
\ref{sec:discussion} describes the implications of these results. We use AB magnitudes and assume 
a \citet{Chabrier03} stellar initial mass function (IMF) with $(H_{0}, \Omega_{M}, \Omega_{\Lambda}) =
(70~{\rm km~s^{-1}~Mpc^{-1}}, 0.3, 0.7)$ throughout.


%------------------------------------------------------------------------------------------------------------------------------------------
%------------------------------------------------------------------------------------------------------------------------------------------

\section{Data}
\label{sec:data}

\subsection{Master sample}
\label{sec:master}

This experiment is based on the {\it Carnegie Spitzer IMACS Survey} \citep[CSI;][]{Kelson14a}. CSI provides
Magellan-IMACS Low- and Uniform-Dispersion Prism spectroscopy (\CITE) for objects with {\it Spitzer} 
$[3.5]\leq21$ in {\bfr XXX sq.~deg.} from {\bfr THESE FIELDS}. Combined with supplemental 
$ugrizJK_{s}$ photometry from the NEWFIRM archive (\CITE) and Canada-France-Hawai`i Telescope 
Legacy Survey (CFHTLS; \CITE), these data were used to derive flexible SFHs for each galaxy as part of 
the redshift estimation process. The sample is complete to $\logM\sim10.3$ at $z\sim0.7$.
The spectral resolution of the prisms varies from $R\sim{\bfr XXX}$ to $R\sim{\bfr YYY}$ at
{\bfr wavelengths}, about {\bfr THIS MUCH WORSE} than the Sloan Digital Sky Survey \citep{York00}.

\citet{Dressler16, Dressler18} examine the CSI SFHs in detail. \citet{Dressler18} provides a thorough treatment 
of CSI SFH quality in its Appendix. We defer the reader to those texts for that information, but briefly review 
the SFH inference process here. {\it None of these details are important in the context of the experiment we 
detail below}, which should be repeated using other approaches.

The CSI spectrophotometry was using 5 precomputed SEDs based on SFHs with constant star formation 
rates (SFRs) spanning:
\bitem
	\item 0.0 to 0.2 Gyr; %prior to $\tobs$;
	\item 0.2 to 0.5 Gyr; %prior to $\tobs$;
	\item 0.5 to 1.0 Gyr; %prior to $\tobs$;
	\item 1.0 to 2.0 Gyr prior to $\tobs$;
	\item either 2.0 Gyr prior to $\tobs$ or $z\simeq 3$ to $z=5$ (1 Gyr),
\eitem
where $\tobs$ corresponds to the object's redshift. The data determine the mode of the oldest bin.
The median redshift of the samples studied in \citet{Dressler16, Dressler18} is $z\sim0.7$, or $\tobs\sim7$ Gyr.

Each of the above SEDs was allowed to take an independent $A_{V}$ (assuming a \citealt{Calzetti00} 
extinction law) but not metallicity. The latter was inferred using a prior peaked at $Z=Z_{\odot}$. As such,
the predicted spectra do not capture the likely enrichment history of any object 
\citep[cf.][]{Pacifici12, Morishita19}---a fact to bear in mind as we proceed.

The SEDs are generated using the Flexible Stellar Population Synthesis code \citep{ConroyGunnWhite09}
assuming default abundance patterns.

\subsection{High-resolution spectroscopy}
\label{sec: hiRes}



%------------------------------------------------------------------------------------------------------------------------------------------
%------------------------------------------------------------------------------------------------------------------------------------------

\section{Confronting predictions with high-resolution data}
\label{sec:results}

\subsection{Systematic errors}
\label{sec:systematics}


%------------------------------------------------------------------------------------------------------------------------------------------
%------------------------------------------------------------------------------------------------------------------------------------------

\section{Implications}
\label{sec:discussion}

\subsection{Where is the information going?}
\label{sec:redshifts}

Redshifts.

%------------------------------------------------------------------------------------------------------------------------------------------
%------------------------------------------------------------------------------------------------------------------------------------------

\section{Summary}
\label{sec:summary}

Foo.\\

\noindent\facilities\ Magellan/IMACS

\noindent\software\ IDL (Coyote libraries; \url{http://www.idlcoyote.com/}), python (\texttt{CarPy}).

%------------------------------------------------------------------------------------------------------------------------------------------
%------------------------------------------------------------------------------------------------------------------------------------------

\section*{Acknowledgements}


%------------------------------------------------------------------------------------------------------------------------------------------
%------------------------------------------------------------------------------------------------------------------------------------------

%%%%%%%%%%%%%%%%%%%% REFERENCES %%%%%%%%%%%%%%%%%%

% The best way to enter references is to use BibTeX:

\bibliographystyle{mnras}
\bibliography{/Users/labramson/lit} % if your bibtex file is called example.bib


% Alternatively you could enter them by hand, like this:
% This method is tedious and prone to error if you have lots of references
%\begin{thebibliography}{99}
%\bibitem[\protect\citeauthoryear{Author}{2012}]{Author2012}
%Author A.~N., 2013, Journal of Improbable Astronomy, 1, 1
%\bibitem[\protect\citeauthoryear{Others}{2013}]{Others2013}
%Others S., 2012, Journal of Interesting Stuff, 17, 198
%\end{thebibliography}

%%%%%%%%%%%%%%%%%%%%%%%%%%%%%%%%%%%%%%%%%%%%%%%%%%

%%%%%%%%%%%%%%%%% APPENDICES %%%%%%%%%%%%%%%%%%%%%

%\appendix
%\label{sec:appendix}

%%%%%%%%%%%%%%%%%%%%%%%%%%%%%%%%%%%%%%%%%%%%%%%%%%


% Don't change these lines
\bsp	% typesetting comment
\label{lastpage}
\end{document}

% End of mnras_template.tex