% mnras_template.tex
%
% LaTeX template for creating an MNRAS paper
%
% v3.0 released 14 May 2015
% (version numbers match those of mnras.cls)
%
% Copyright (C) Royal Astronomical Society 2015
% Authors:
% Keith T. Smith (Royal Astronomical Society)

% Change log
%
% v3.0 May 2015
%    Renamed to match the new package name
%    Version number matches mnras.cls
%    A few minor tweaks to wording
% v1.0 September 2013
%    Beta testing only - never publicly released
%    First version: a simple (ish) template for creating an MNRAS paper

%%%%%%%%%%%%%%%%%%%%%%%%%%%%%%%%%%%%%%%%%%%%%%%%%%
% Basic setup. Most papers should leave these options alone.
\documentclass[a4paper,fleqn,usenatbib]{mnras}

% MNRAS is set in Times font. If you don't have this installed (most LaTeX
% installations will be fine) or prefer the old Computer Modern fonts, comment
% out the following line
\usepackage{newtxtext,newtxmath}
% Depending on your LaTeX fonts installation, you might get better results with one of these:
%\usepackage{mathptmx}
%\usepackage{txfonts}

% Use vector fonts, so it zooms properly in on-screen viewing software
% Don't change these lines unless you know what you are doing
\usepackage[T1]{fontenc}
%\usepackage{ae,aecompl}


%%%%% AUTHORS - PLACE YOUR OWN PACKAGES HERE %%%%%

% Only include extra packages if you really need them. Common packages are:
\usepackage{graphicx}	% Including figure files
\usepackage{amsmath}	% Advanced maths commands
\usepackage{amssymb}	% Extra maths symbols
\usepackage{color}

%%%%%%%%%%%%%%%%%%%%%%%%%%%%%%%%%%%%%%%%%%%%%%%%%%

%%%%% AUTHORS - PLACE YOUR OWN COMMANDS HERE %%%%%

\newcommand{\Mpc}{{\rm Mpc}}
\newcommand{\km}{{\rm km}}
\newcommand{\kpc}{{\rm kpc}}
\newcommand{\pc}{\ {\rm pc}}
\newcommand{\kms}{{\rm km}\,{\rm s}^{-1}}
\newcommand{\yr}{{\rm yr}}
\newcommand{\Msun}{{\rm M}_\odot}
\newcommand{\Mstel}{M_\ast}
\newcommand{\logM}{\log\Mstel/\Msun}
\newcommand{\LCDM}{$\Lambda$CDM}
\newcommand{\resp}{respectively}
\newcommand{\bfr}{\bf\color{red}}
\newcommand{\bfb}{\color{myblue}}
\newcommand{\bfnull}{\color{black}}
\newcommand{\bfc}{\sf\color{myblue}}
\newcommand{\bfp}{\bf\color{magenta}}
\newcommand{\ssfr}{{\rm sSFR}}
\newcommand{\sfr}{{\rm SFR}}
\newcommand{\zphot}{z_{\rm phot}}
\newcommand{\zspec}{z_{\rm spec}}

\newcommand{\beq}{\begin{equation}}
\newcommand{\eeq}{\end{equation}}
\newcommand{\bitem}{\begin{itemize}}
\newcommand{\eitem}{\end{itemize}}
\newcommand{\benum}{\begin{enumerate}}
\newcommand{\eenum}{\end{enumerate}}

\mathchardef\mhyphen="2D

\newcommand{\ntot}{{\bfr XXX}} % total objects in high-res sample
\newcommand{\midz}{{\bfr ZZZ}} % their median redshift

\newcommand{\CITE}{{\bfr CITE}}
\newcommand{\facilities}{{\it Facilities:}}
\newcommand{\software}{{\it Software:}}

% Please keep new commands to a minimum, and use \newcommand not \def to avoid
% overwriting existing commands. Example:
%\newcommand{\pcm}{\,cm$^{-2}$}	% per cm-squared

%%%%%%%%%%%%%%%%%%%%%%%%%%%%%%%%%%%%%%%%%%%%%%%%%%

%%%%%%%%%%%%%%%%%%% TITLE PAGE %%%%%%%%%%%%%%%%%%%

% Title of the paper, and the short title which is used in the headers.
% Keep the title short and informative.
\title[Who needs spectra?]{Spectral resolution is not important for modeling galaxy growth}

% The list of authors, and the short list which is used in the headers.
% If you need two or more lines of authors, add an extra line using \newauthor
\author[Abramson, Kelson, \& Dressler]{Louis E.~Abramson$^{1}$\thanks{E-mail: \href{mailto:labramson@carnegiescience.edu}{labramson@carnegiescience.edu}},
Daniel D.~Kelson$^{1}$,
and Alan Dressler$^{1}$
\\
\\
% List of institutions
$^1$	Carnegie Observatories, 813 Santa Barbara Street, Pasadena, CA 91101, USA\\
}

% These dates will be filled out by the publisher
\date{Accepted XXX. Received YYY; in original form ZZZ}
%\date{Submitted to {\it MNRAS} 31 May 2019}

% Enter the current year, for the copyright statements etc.
\pubyear{2020}

% Don't change these lines
\begin{document}
\label{firstpage}
\pagerange{\pageref{firstpage}--\pageref{lastpage}}
\maketitle

% Abstract of the paper
\begin{abstract}

	High resolution spectra are assumed to yield better constraints on galaxy star formation
	histories (SFHs) than low resolution spectra or photometry. We test this statement 
	by comparing actual $R\sim{\bfr 800}$ spectroscopy to high-resolution model predictions 
	based on galaxy SFHs inferred from only $ugrizJK_{s}$ photometry and $R\sim25$ 
	rest-optical spectra. From \ntot\ galaxies, we find the median difference between predicted 
	and inferred Lick indices to be $\leq1\%$ for all indices except the Blamer lines---whose 
	divergences are readily explained by unmodeled emission---and the CN$_{1}$ and G4300 
	features in passive galaxies, which diverge by 2\% and 1.5\%, \resp. {\bfr $\chi^{2}$ stuff.} 
	Absent a scientific motivation to measure such features to better precision, we therefore
	find no utility in adding high resolution spectroscopy as an SFH modeling constraint, at least 
	when the models are capable of capturing the intrinsic diversity of real growth trajectories. 
	Our results cast doubt on the extent to which projects like the {\it James Webb Space Telescope}
	will enhance our understanding of core questions in galaxy evolution and suggest that progress 
	requires new tactics as much as new data.
%	By increasing the number of data points without adding meaningful physical details, such data 
%	may in fact harm our understanding by unrealistically shrinking SFH uncertainties. 
	%, if not the fundamental
%	utility of the project of inferring SFHs from galaxy-level data.

\end{abstract}
%{\it Hubble Frontier Fields} and {\it Cluster Lensing and Supernova with Hubble}

% Select between one and six entries from the list of approved keywords.
% Don't make up new ones.
\begin{keywords}
	galaxies: surveys --- galaxies: spectroscopy --- spectroscopy: techniques
\end{keywords}

%%%%%%%%%%%%%%%%%%%%%%%%%%%%%%%%%%%%%%%%%%%%%%%%%%

%%%%%%%%%%%%%%%%% BODY OF PAPER %%%%%%%%%%%%%%%%%%

\section{Introduction}
\label{sec:intro}

A central ambition of the study of galaxy evolution is understanding stellar mass growth; i.e., galaxy
star formation histories (SFHs). Spectral energy distributions (SEDs) are the key data in this work because 
they can be decomposed into combinations of distinct stellar populations of known ages. The resulting 
coefficients yield the amount of stellar mass a galaxy is inferred to have formed at the lookback time 
corresponding to each population's age.

Stellar populations have different but not orthogonal SEDs. As such, the above decompositions are 
usually degenerate. Such degeneracies are compounded by age-independent effects like metallicity and 
dust reddening. 

To alleviate those degeneracies, high resolution spectra ($R\sim1000$s) are often used in addition to 
galaxy broadband colors in the model fitting. The hope is that the details of absorption lines will potentially 
increase the contrast between stellar subpopulations, constrain metallicity, and so yield more accurate 
age/mass coefficients. The utility of these data is usually taken as axiomatic, but it is also readily testable. 

This paper presents an experiment that shows there is very little information in high resolution spectra
that enhance constraints on galaxy SFHs compared to inferences based on a combination of photometry and 
very low resolution ($R\lesssim100$s) prism spectra.\footnote{A future paper will extend this statement 
to photometry based inferences alone.} We perform this experiment by using precomputed SFH inferences
based on low resolution SEDs for a set of \ntot\ systems at $\langle z\rangle=\midz$ to produce predictions 
of each galaxy's high-resolution spectrum, and comparing these predictions to actual high resolution 
observations taken post-facto. With the exception of the Balmer lines---whose divergence from predictions 
is readily ascribable to emission line infilling---we find differences to be of order {\bfr whatever they are}, 
suggesting {\bfr whatever we say they do}.

Below, Section \ref{sec:data} describes the data on which our experiment is based, Section \ref{sec:results} 
shows the comparisons between our high resolution spectral predictions and the high resolution data, and
Section \ref{sec:discussion} describes the implications of these results. We use AB magnitudes and assume 
a \citet{Chabrier03} stellar initial mass function (IMF) with $(H_{0}, \Omega_{M}, \Omega_{\Lambda}) =
(70~{\rm km~s^{-1}~Mpc^{-1}}, 0.3, 0.7)$ throughout.


%------------------------------------------------------------------------------------------------------------------------------------------
%------------------------------------------------------------------------------------------------------------------------------------------

\section{Data}
\label{sec:data}

\subsection{Master sample}
\label{sec:master}

This experiment is based on the {\it Carnegie Spitzer IMACS Survey} \citep[CSI;][]{Kelson14a}. CSI provides
Magellan-IMACS Low- and Uniform-Dispersion Prism spectroscopy for objects with {\it Spitzer} 
$[3.5]\leq21$ in {\bfr XXX sq.~deg.} from {\bfr THESE FIELDS}. Combined with supplemental 
$ugrizJK_{s}$ photometry from the NEWFIRM archive (\CITE) and Canada-France-Hawai`i Telescope 
Legacy Survey (CFHTLS; \CITE), these data were used to derive flexible SFHs for each galaxy as part of 
the redshift estimation process. The sample is complete to $\log\Mstel\sim10.3$ at $z\sim0.7$.
The spectral resolution of the prisms varies from $R\sim{\bfr XXX}$ to $R\sim{\bfr YYY}$ at
{\bfr wavelengths}, about {\bfr THIS MUCH WORSE} than the Sloan Digital Sky Survey \citep{York00}.

\citet{Dressler16, Dressler18} examine the CSI SFHs in detail. \citet{Dressler18} provides a thorough treatment 
of SFH inference quality in its Appendix as assessed via simulations and comparisons of objects with repeat 
observations. We defer the reader to those texts for that information, but briefly review the SFH inference 
process here. {\it None of these details are important in the context of the experiment we detail below}, which 
should be repeated using other approaches.

The CSI spectrophotometry was modeled by assuming 5 blocks SFHs spanned 

\subsection{High-resolution spectroscopy}
\label{sec: hiRes}



%------------------------------------------------------------------------------------------------------------------------------------------
%------------------------------------------------------------------------------------------------------------------------------------------

\section{Confronting predictions with high-resolution data}
\label{sec:results}


%------------------------------------------------------------------------------------------------------------------------------------------
%------------------------------------------------------------------------------------------------------------------------------------------

\section{Implications}
\label{sec:discussion}


%------------------------------------------------------------------------------------------------------------------------------------------
%------------------------------------------------------------------------------------------------------------------------------------------

\section{Summary}
\label{sec:summary}

Foo.\\

\noindent\facilities\ Magellan/IMACS

\noindent\software\ IDL (Coyote libraries; \url{http://www.idlcoyote.com/}), python (\texttt{CarPy}).

%------------------------------------------------------------------------------------------------------------------------------------------
%------------------------------------------------------------------------------------------------------------------------------------------

\section*{Acknowledgements}


%------------------------------------------------------------------------------------------------------------------------------------------
%------------------------------------------------------------------------------------------------------------------------------------------

%%%%%%%%%%%%%%%%%%%% REFERENCES %%%%%%%%%%%%%%%%%%

% The best way to enter references is to use BibTeX:

\bibliographystyle{mnras}
\bibliography{/Users/labramson/lit} % if your bibtex file is called example.bib


% Alternatively you could enter them by hand, like this:
% This method is tedious and prone to error if you have lots of references
%\begin{thebibliography}{99}
%\bibitem[\protect\citeauthoryear{Author}{2012}]{Author2012}
%Author A.~N., 2013, Journal of Improbable Astronomy, 1, 1
%\bibitem[\protect\citeauthoryear{Others}{2013}]{Others2013}
%Others S., 2012, Journal of Interesting Stuff, 17, 198
%\end{thebibliography}

%%%%%%%%%%%%%%%%%%%%%%%%%%%%%%%%%%%%%%%%%%%%%%%%%%

%%%%%%%%%%%%%%%%% APPENDICES %%%%%%%%%%%%%%%%%%%%%

%\appendix
%\label{sec:appendix}

%%%%%%%%%%%%%%%%%%%%%%%%%%%%%%%%%%%%%%%%%%%%%%%%%%


% Don't change these lines
\bsp	% typesetting comment
\label{lastpage}
\end{document}

% End of mnras_template.tex