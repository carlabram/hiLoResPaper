% mnras_template.tex
%
% LaTeX template for creating an MNRAS paper
%
% v3.0 released 14 May 2015
% (version numbers match those of mnras.cls)
%
% Copyright (C) Royal Astronomical Society 2015
% Authors:
% Keith T. Smith (Royal Astronomical Society)

% Change log
%
% v3.0 May 2015
%    Renamed to match the new package name
%    Version number matches mnras.cls
%    A few minor tweaks to wording
% v1.0 September 2013
%    Beta testing only - never publicly released
%    First version: a simple (ish) template for creating an MNRAS paper

%%%%%%%%%%%%%%%%%%%%%%%%%%%%%%%%%%%%%%%%%%%%%%%%%%
% Basic setup. Most papers should leave these options alone.
\documentclass[a4paper,fleqn,usenatbib]{mnras}

% MNRAS is set in Times font. If you don't have this installed (most LaTeX
% installations will be fine) or prefer the old Computer Modern fonts, comment
% out the following line
\usepackage{newtxtext,newtxmath}
% Depending on your LaTeX fonts installation, you might get better results with one of these:
%\usepackage{mathptmx}
%\usepackage{txfonts}

% Use vector fonts, so it zooms properly in on-screen viewing software
% Don't change these lines unless you know what you are doing
\usepackage[T1]{fontenc}
%\usepackage{ae,aecompl}


%%%%% AUTHORS - PLACE YOUR OWN PACKAGES HERE %%%%%

% Only include extra packages if you really need them. Common packages are:
\usepackage{graphicx}	% Including figure files
\usepackage{amsmath}	% Advanced maths commands
\usepackage{amssymb}	% Extra maths symbols
\usepackage{color}

%%%%%%%%%%%%%%%%%%%%%%%%%%%%%%%%%%%%%%%%%%%%%%%%%%

%%%%% AUTHORS - PLACE YOUR OWN COMMANDS HERE %%%%%

\newcommand{\Mpc}{{\rm Mpc}}
\newcommand{\km}{{\rm km}}
\newcommand{\kpc}{{\rm kpc}}
\newcommand{\pc}{\ {\rm pc}}
\newcommand{\kms}{{\rm km}\,{\rm s}^{-1}}
\newcommand{\yr}{{\rm yr}}
\newcommand{\Msun}{{\rm M}_\odot}
\newcommand{\Mstel}{M_\ast}
\newcommand{\logM}{\log\Mstel/\Msun}
\newcommand{\LCDM}{$\Lambda$CDM}
\newcommand{\resp}{respectively}
\newcommand{\bfr}{\bf\color{red}}
\newcommand{\bfb}{\color{myblue}}
\newcommand{\bfnull}{\color{black}}
\newcommand{\bfc}{\sf\color{myblue}}
\newcommand{\bfp}{\bf\color{magenta}}
\newcommand{\ssfr}{{\rm sSFR}}
\newcommand{\sfr}{{\rm SFR}}
\newcommand{\tobs}{t_{\rm obs}}
\newcommand{\zphot}{z_{\rm phot}}
\newcommand{\zspec}{z_{\rm spec}}

\newcommand{\beq}{\begin{equation}}
\newcommand{\eeq}{\end{equation}}
\newcommand{\bitem}{\begin{itemize}}
\newcommand{\eitem}{\end{itemize}}
\newcommand{\benum}{\begin{enumerate}}
\newcommand{\eenum}{\end{enumerate}}

\mathchardef\mhyphen="2D

\newcommand{\ntot}{{\bfr XXX}} % total objects in high-res sample
\newcommand{\midz}{{\bfr ZZZ}} % their median redshift

\newcommand{\CITE}{{\bfr CITE}}
\newcommand{\facilities}{{\it Facilities:}}
\newcommand{\software}{{\it Software:}}

% Please keep new commands to a minimum, and use \newcommand not \def to avoid
% overwriting existing commands. Example:
%\newcommand{\pcm}{\,cm$^{-2}$}	% per cm-squared

%%%%%%%%%%%%%%%%%%%%%%%%%%%%%%%%%%%%%%%%%%%%%%%%%%

%%%%%%%%%%%%%%%%%%% TITLE PAGE %%%%%%%%%%%%%%%%%%%

% Title of the paper, and the short title which is used in the headers.
% Keep the title short and informative.
\title[Who needs spectra?]{Spectral resolution is not important for modeling galaxy growth}

% The list of authors, and the short list which is used in the headers.
% If you need two or more lines of authors, add an extra line using \newauthor
\author[Abramson, Kelson, \& Dressler]{Louis E.~Abramson$^{1}$\thanks{E-mail: \href{mailto:labramson@carnegiescience.edu}{labramson@carnegiescience.edu}},
Daniel D.~Kelson$^{1}$,
and Alan Dressler$^{1}$
\\
\\
% List of institutions
$^1$	Carnegie Observatories, 813 Santa Barbara Street, Pasadena, CA 91101, USA\\
}

% These dates will be filled out by the publisher
\date{Accepted XXX. Received YYY; in original form ZZZ}
%\date{Submitted to {\it MNRAS} 31 May 2019}

% Enter the current year, for the copyright statements etc.
\pubyear{2020}

% Don't change these lines
\begin{document}
\label{firstpage}
\pagerange{\pageref{firstpage}--\pageref{lastpage}}
\maketitle

% Abstract of the paper
\begin{abstract}

	We compare actual $R\sim{\bfr 800}$ spectroscopy to model predictions based on galaxy 
	star formation histories (SFHs) inferred from much lower resolution data: $ugrizJK_{s}$ 
	photometry and $R\sim25$ rest-optical prism spectra. From \ntot\ systems, we find a median 
	difference of $\leq$1\% between all predicted and measured absorption features in the Lick index 
	bandpasses except the Blamer lines---explainable by unmodeled emission---and Ca4227 and 
	Fe5270 in {\it UVJ}-classified passive galaxies, which are 1.7\%--2.5\% weaker than expected. 
	{\bfr $\chi^{2}$ stuff.} As such, absent a Ca-- or Fe--age prior accurate to the 2\% level---whose 
	empirical apprehension is itself a motivation for SED fitting---we find no utility in adding 
	high resolution spectroscopy as an SFH modeling constraint, at least when using models that capture 
	the intrinsic diversity of real growth trajectories. Our results cast doubt on the extent to which 
	spectra from the {\it James Webb Space Telescope} will enhance our understanding of galaxy 
	growth and suggest that progress requires new tactics as much as new data.
%	or a $\sigma_{v}-\Mstel$ prior with less than $\sim$0.2	dex of scatter
%	By increasing the number of data points without adding meaningful physical details, such data 
%	may in fact harm our understanding by unrealistically shrinking SFH uncertainties. 
	%, if not the fundamental
%	utility of the project of inferring SFHs from galaxy-level data.

\end{abstract}

% Select between one and six entries from the list of approved keywords.
% Don't make up new ones.
\begin{keywords}
	galaxies: spectroscopy
\end{keywords}

%%%%%%%%%%%%%%%%%%%%%%%%%%%%%%%%%%%%%%%%%%%%%%%%%%

%%%%%%%%%%%%%%%%% BODY OF PAPER %%%%%%%%%%%%%%%%%%

\section{Introduction}
\label{sec:intro}

A central ambition of the study of galaxy evolution is to understand stellar mass growth; i.e., galaxy
star formation histories (SFHs). Spectral energy distributions (SEDs) are the key data in this work because 
they can be decomposed into combinations of distinct stellar subpopulations of known ages. The resulting 
coefficients yield the amount of stellar mass a galaxy is inferred to have formed at the lookback time 
corresponding to each subpopulation's age.
	
Different stellar subpopulations have different but not orthogonal SEDs. As such, galaxy decompositions 
are degenerate. Of course, those degeneracies are compounded by age-independent effects like metallicity 
and dust. 

High resolution spectra ($R\sim500$--5000) are often used to alleviate those degeneracies in SFH model 
fitting. The hope is that the absorption features that emerge in those data will increase the contrast 
between constituent stellar subpopulations, constrain metallicities, and yield more accurate 
age/mass coefficients. The utility of these data is usually taken as axiomatic, but it is also testable. 

Here we present experiment that shows there is in fact little information in high resolution spectra
that enhance constraints on galaxy SFHs compared to inferences based on a combination of broadband 
photometry and low resolution ($R\sim25$) prism spectra.\footnote{A future paper will extend this statement 
to pure photometry-based inferences.} We use precomputed SFH inferences based on such low resolution 
SEDs for a set of \ntot\ systems at $\langle z\rangle=\midz$ to produce predictions 
of each galaxy's high-resolution spectrum. We then compare those predictions to actual high resolution 
($R\sim800$) observations taken post-facto. With the exception of the Balmer lines---whose divergence 
from predictions is readily ascribable to emission line infilling---we find differences to be of order 
{\bfr whatever they are}, suggesting {\bfr whatever we say they do}.

Section \ref{sec:data} describes the data on which our experiment is based, Section \ref{sec:results} 
shows the comparisons between our spectral predictions and the high resolution data, and Section
\ref{sec:discussion} describes the implications of these results. We use AB magnitudes and assume 
a \citet{Chabrier03} stellar initial mass function (IMF) with $(H_{0}, \Omega_{M}, \Omega_{\Lambda}) =
(70~{\rm km~s^{-1}~Mpc^{-1}}, 0.3, 0.7)$ throughout.


%------------------------------------------------------------------------------------------------------------------------------------------
%------------------------------------------------------------------------------------------------------------------------------------------

\section{Data}
\label{sec:data}

\subsection{Master sample}
\label{sec:master}

This experiment is based on the {\it Carnegie Spitzer IMACS Survey} \citep[CSI;][]{Kelson14a}. CSI provides
Magellan-IMACS Low- and Uniform-Dispersion Prism spectroscopy (\CITE) for objects with {\it Spitzer} 
$[3.5]\leq21$ in {\bfr XXX sq.~deg.} from {\bfr THESE FIELDS}. Combined with supplemental 
$ugrizJK_{s}$ photometry from the NEWFIRM archive (\CITE) and Canada-France-Hawai`i Telescope 
Legacy Survey (CFHTLS; \CITE), these data were used to derive flexible SFHs for each galaxy as part of 
the redshift estimation process. The sample is complete to $\logM\sim10.3$ at $z\sim0.7$.
The spectral resolution of the prisms varies from $R\sim{\bfr XXX}$ to $R\sim{\bfr YYY}$ at
{\bfr wavelengths}, about {\bfr THIS MUCH WORSE} than the Sloan Digital Sky Survey \citep{York00}.

\citet{Dressler16, Dressler18} examine the CSI SFHs in detail. \citet{Dressler18} provides a thorough treatment 
of CSI SFH quality in its Appendix. We defer the reader to those texts for that information, but briefly review 
the SFH inference process here. {\it None of these details are important in the context of the experiment we 
detail below}, which should be repeated using other approaches.

The CSI spectrophotometry was using 5 precomputed SEDs based on SFHs with constant star formation 
rates (SFRs) spanning:
\bitem
	\item 0.0 to 0.2 Gyr prior to $\tobs$;
	\item 0.2 to 0.5 Gyr prior to $\tobs$;
	\item 0.5 to 1.0 Gyr prior to $\tobs$;
	\item 1.0 to 2.0 Gyr prior to $\tobs$;
	\item 2.0 Gyr prior to $\tobs$ to $z=5$;
\eitem
where $\tobs$ corresponds to the object's redshift and $z=5$ is taken as the onset of star formation. If 
the data prefer, the oldest bin can also take the form of a 1 Gyr top hat starting at $z=5$.
The median redshift of the samples studied in \citet{Dressler16, Dressler18} is $z\sim0.7$, or $\tobs\sim7$ Gyr.

Each of the above SEDs was allowed to take an independent $A_{V}$ (assuming a \citealt{Calzetti00} 
extinction law) but not metallicity. The latter was inferred using a prior peaked at $Z=Z_{\odot}$. As such,
the predicted spectra do not capture the likely enrichment history of any object 
\citep[cf.][]{Pacifici12, Morishita19}---a fact to bear in mind as we proceed.

All spectral templates were generated using Flexible Stellar Population Synthesis 
\citep[FSPS;][]{ConroyGunnWhite09} assuming default abundance patterns. When inferring the SFHs, 
these models---5 mass amplitudes + 5 $A_{V}$s + 1 metallicity + 1 redshift + {\bfr 1} spectrophotometry fluxing 
factor + {\bfr N} emission line amplitudes = {\bfr XXX} free parameters---were typically constrained by 
7 + {\bfr $\sim$100} photometric + spectral datapoints. The spectral models studied below are were 
regenerated using the best-fit SFH parameters at $R\sim800$ in the rest optical. Uncertainties are tabulated
for the SFH bins but {\it not} propagated to the high resolution model spectra. As they therefore rely purely on 
data uncertainties, all comparisons below should be regarded as conservative.

\subsection{High-resolution spectroscopy}
\label{sec: hiRes}

{\bfr We took high resolution data with IMACS. Dan will tell you all about it, including how many objects, 
and a brief description of $S/N$/data quality/etc.}

%------------------------------------------------------------------------------------------------------------------------------------------
%------------------------------------------------------------------------------------------------------------------------------------------

\section{Confronting predictions with high-resolution data}
\label{sec:results}

\subsection{Systematic errors}
\label{sec:systematics}


%------------------------------------------------------------------------------------------------------------------------------------------
%------------------------------------------------------------------------------------------------------------------------------------------

\section{Implications}
\label{sec:discussion}

Spectral features observable in costly, high-resolution spectroscopy are readily predicted by 
models fit to cheaper broadband photometry and prism spectra. As such, obtaining the former 
seems irrelevant for investigators interested in inferring galaxy SFHs.

This fact prompts three questions: (1) Why? (2) What are the extra pixels doing? (3) What does
this mean for the future? All of these relate to the difference between formal and meaningful 
information content.

\subsection{Spectral pixels are correlated}
\label{sec:pixels}

Driving the outcomes we find is the reality that most of the pixels in high resolution 
spectra are physically correlated, even if they are mathematically independent. This is obvious to 
some extent---Balmer line depths track $B-V$ or D4000---but we find that practically all of the 
line information is determined by some (unknown) combination of the local and remote 
continuum. These physical correlations should be mathematically accounted for in the SED 
modeling to avoid making statements of unreasonable certainty.

The nature of this correlation is the mathematical key to understanding the SFH$\,\mapsto\,$SED 
mapping, and therefore our ability to invert the latter to get at the former; i.e., execute an empirical 
study of galaxy evolution. Machine learning may be the way to illuminate this mapping, whose concrete 
output would be a hypersurface characterizing the precision to which $\Mstel(t)$ is informatically 
accessible as a function of data---$S/N$, SED sampling---and galaxy properties---$z$, $\ssfr$, $Z$, 
$\sigma$, $A_{V}$, environment, morphology. 

A first step in the above endeavor is to invert the experiment we performed here. We suspect 
that wavelength baseline is the underlying key to SED fitting. To test this, one could derive
SFHs based on fits to the high resolution spectra presented here and try to predict the broadband
fluxes well outside the rest optical. Examining Figure 12 of \citet{Abramson20}, fits to
low resolution rest-optical spectra like those from CSI used here do not predict, e.g., infrared
broadband fluxes to the level the high resolution lines are reproduced here. The outcome of
this test using high resolution spectra as a starting point would inform us as to the relative
priority of broad baseline photometry vs deep spectroscopy in the study of galaxy growth. 
{\bfr This is crap; you always need a redshift and I think ultimately that's the trick.}

In the immediate term, the upshot is that suitably sampled SEDs need never be well sampled, at least
not to understand mass growth (Section \ref{sec:redshifts}). What ``suitable'' means would be determined 
by the above investigation, but clearly CSI meets or exceeds that threshold at $z\lesssim1$.\footnote{A future 
paper will show that the 26-band UltraVista filter set \citep{Muzzin13} probably also does as well at 
$z\sim0.4$.} Of course, this implies that many current (and future; Section \ref{sec:future}) data sets 
are oversampled, and that it is inadvisable to use those for SFH modeling while assuming that each 
pixel adds one degree of freedom. Adding systematics that correlate pixels is a minimum requirement 
to reinflate the resultant SFH uncertainties, but a floor will ultimately be necessary. It is very possible 
that we have already reached that floor.
% in terms of the number/span of bandpasses required to 
%predict SFHs to a given precision as a function of redshift 
%though it is very different from the question of ``What is a galaxy's SFH?'' 

\subsection{Where is the information going?}
\label{sec:redshifts}

The explicit science motivator for this study was inferring SFH, and, to a lesser extent, galaxy 
enrichment histories, which we have shown are consistent with flat to {\bfr XXX\%}. However, 
these are not the only applications for high resolution spectroscopy, and it is in other domains 
where the extra pixels indeed play a role.

Principle among these in terms of SFH reconstruction is the determination of redshift. At high 
resolution, the available number
of constraining features is much higher than in the CSI data. {\bfr Something about the average
redshift offset we find, the need to allow that to re-float when moving the predictions, whether
that level of jiggle was appropriately marginalzied over in the original SFH inference, and whether
that mattered.} This means that redshift--SFH covariance is stronger than it needs to be, inflating 
the uncertainties on the latter. Furthermore, the added uncertainty comes in key places: At CSI's 
spectral resolution, [\ion{O}{ii}] $\lambda 3727$ is almost indistinguishable from the Balmer break. 
While the break has substantial leverage on the SFH, [\ion{O}{ii}] has none. A failure to resolve these 
may bias the SFH by implying a larger amount of A stars than is real. While we do not find this to 
be a critical problem ({\bfr see Figures XXX}), it does rely on accurate emission line modeling, 
which may be difficult in the case of ever more ubiquitous slitless spectroscopy (from HST, JWST). 

Secondary in the context of SFH reconstruction is in assessing stellar velocity dispersions. Each 
additional sampling of a line profile increases the certainty on that parameter (at fixed $S/N$). 
This certainty is useful because velocity dispersions are color-independent mass proxies and are 
thus {\it truly} orthogonal to stellar subpopulation SEDs into which one is decomposing a galaxy SED.
Velocity dispersions might thus provide strict mass limits capable of breaking some of the degeneracies 
mentioned in Section \ref{sec:intro}: If there existed a suitably tight relation between $\sigma$
and $\Mstel$, one could apply it as a prior in the SED fitting to constrain the amount of faint old 
stars that would imprint on the dispersion but perhaps not be detectable in the SED.

Unfortunately, at {\bfr 0.2 dex?}, that relation is probably not tight enough to be decisive. Further,
allowances should be made that---due, e.g., to inside-out growth (\CITE), the existence of thick disks, 
etc.---each stellar subpopulation has a different velocity dispersion, such the same degeneracies in
decomposing the global $\sigma$ constraint into pieces arise as did with the SED. Both of these 
effects would have to be marginalized over. While we do not know precisely how these effects 
would modulate the utility of $\sigma$ as an SFH constraint, we would predict them to 
substantially erode it.

{\bfr A paragraph on the trivial cases of emission lines and IMF?}

\if 0
The trivial instance of this is in measuring emission lines, where resolution is needed to 
identify, e.g., embedded Balmer lines and detect subtle features like [\ion{O}{iii}] 
$\lambda4363$ ({\bfr cite Sanders}). Especially at low $S/N$, each pixel definitely counts
for these applications, and estimates of, e.g., nebular extinction, ISM metallicity, or outflow
properties will improve with every additional datum.

A yet more pertinent use would be to constrain the IMF \citep{Conroy12}. While doing so 
would have no bearing on the relative accuracies of SFH inferences, it would allow all stellar 
masses to be placed on a firmer absolute scale, which should enhance these objects power to test 
simulations or first-principles theoretical predictions. %Of course, if the IMF is allowed to vary with 
%time, one encounters the same problems as with $\sigma$.
%Redshifts, velocity dispersions, emission line details.
\fi

\subsection{Future Work}
\label{sec:future}

{\it Don't use JWST to take high resolution spectroscopy if you're interested in stellar mass growth!}

%------------------------------------------------------------------------------------------------------------------------------------------
%------------------------------------------------------------------------------------------------------------------------------------------

\section{Summary}
\label{sec:summary}



\noindent\facilities\ Magellan/IMACS

\noindent\software\ Python (\texttt{CarPy}).%IDL (Coyote libraries; \url{http://www.idlcoyote.com/})

%------------------------------------------------------------------------------------------------------------------------------------------
%------------------------------------------------------------------------------------------------------------------------------------------

\section*{Acknowledgements}


%------------------------------------------------------------------------------------------------------------------------------------------
%------------------------------------------------------------------------------------------------------------------------------------------

%%%%%%%%%%%%%%%%%%%% REFERENCES %%%%%%%%%%%%%%%%%%

% The best way to enter references is to use BibTeX:

\bibliographystyle{mnras}
\bibliography{/Users/labramson/lit} % if your bibtex file is called example.bib


% Alternatively you could enter them by hand, like this:
% This method is tedious and prone to error if you have lots of references
%\begin{thebibliography}{99}
%\bibitem[\protect\citeauthoryear{Author}{2012}]{Author2012}
%Author A.~N., 2013, Journal of Improbable Astronomy, 1, 1
%\bibitem[\protect\citeauthoryear{Others}{2013}]{Others2013}
%Others S., 2012, Journal of Interesting Stuff, 17, 198
%\end{thebibliography}

%%%%%%%%%%%%%%%%%%%%%%%%%%%%%%%%%%%%%%%%%%%%%%%%%%

%%%%%%%%%%%%%%%%% APPENDICES %%%%%%%%%%%%%%%%%%%%%

%\appendix
%\label{sec:appendix}

%%%%%%%%%%%%%%%%%%%%%%%%%%%%%%%%%%%%%%%%%%%%%%%%%%


% Don't change these lines
\bsp	% typesetting comment
\label{lastpage}
\end{document}

% End of mnras_template.tex